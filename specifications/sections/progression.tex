\documentclass[../doc.tex]{subfiles}

\begin{document}
\section{Système de progression}
Un des points qui fait du jeu vidéo le passe-temps
favori de nombreuses personnes, c'est le sentiment de
progression. Le joueur veut sentir que ses efforts sont,
d'une façon ou d'un autre, récompensés autrement que par un
simple écran de victoire. C'est pourquoi nous voulons intégrer
un système de progression qui peut se traduire par un ensemble
d'éléments évoluant au fil des parties et du temps passé à jouer.
\newline\indent
Un exemple très courant et très apprécié des joueurs, ce sont les
\textit{achievements}, ou succès, qui sont débloqués quand le joueur réalise
Xune action spécifique (par exemple, après avoir gagné x parties,
ou après avoir tué X personnes). Un tel système doit aussi s'accompagner d'un mode de sauvegarde d'informations,
car il faut que les données concernant la progression du joueur
soient persistantes, même au redémarrage, pour pouvoir suivre l'évolution au fil de plusieurs sessions de jeu.
\end{document}