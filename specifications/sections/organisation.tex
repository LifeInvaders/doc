\documentclass[../doc.tex]{subfiles}

\begin{document}
\section{Organisation de partie}
L'organisation de partie concerne la création des scripts
et ressources permettant à chaque partie multijoueur de se dérouler suivant un schéma prédéterminé. Cela comprend le début et 
la fin de manche (choix du gagnant, présentation du classement
de fin de partie par exemple), le placement des éléments dynamiques de la scène,
les timers, le comptage des points, l'assignation des cibles...
\\

\indent
Pour offrir la meilleure expérience de jeu possible, cette organisation
doit permettre à chaque manche d'être unique et imprévisible.
Pour cela, nous introduirons de la complexité dans les différentes
mécaniques de jeu. Par exemple, on peut attribuer plus de points
au premier joueur ayant abattu sa cible ou à un joueur ayant
survécu un certain temps sans se faire tuer.
\end{document}