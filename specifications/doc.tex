\documentclass[12pt, letterpaper, twoside]{article}
\usepackage[utf8]{inputenc}%codification of the document

\usepackage{comment}
\usepackage{subfiles}

\title{Panic At Tortuga \thanks{From LifeInvaders Production Team}}

\author{
    KEDJNANE, Harrys\\
    \texttt{harrys.kedjnane@epita.fr}
    \and
    DEVERS, Renaud-Dov\\
    \texttt{renaud-dov.devers@epita.fr}
    \and
    COHEN-SCALI, Julien\\
    \texttt{julien.cohen-scali@epita.fr}
    \and
    DE LA PORTE DES VAUX, Paul\\
    \texttt{paul.de-la-porte-des-vaux@epita.fr}
    
}
% \date{February 2014}

%Here begins the body of the document
\begin{document}

\begin{titlepage}
\maketitle
\end{titlepage}

\begin{center}
Nous allons vous présenter notre projet de S2. Notre équipe a choisi de réaliser un jeu vidéo pour le projet du SUP.
\end{center}

\tableofcontents
\newpage


\section{Présentation du jeu}
\begin{flushleft}
Panic At Tortuga est un jeu multijoueur dans lequel les différents joueurs doivent s'entre-éliminer sur une île.
Le jeu s'inspire de jeux connus : Le premier est le mod "Guess Who ?" où une première équipe doit rechercher et éliminer l'autre équipe qui s'est déguisée en NPC (non-player character) et qui doit imiter les mouvements aléatoires du bot.
% ee\footnote[1]{Lui même inspiré des jeux Spy Party, Hide&Seek et PropHunt}
\end{flushleft}

\section{Game Core}
\subfile{sections/gamecore.tex}

\section{Intelligence Artificielle}
\subfile{sections/ai.tex}

\section{Multijoueur}
\subfile{sections/multiplayer.tex}

\section{Modélisation de la carte}
\subfile{sections/map.tex}


\end{document}

