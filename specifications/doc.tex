\documentclass[french, 12pt, a4paper,twoside]{article}


\usepackage[utf8]{inputenc}
% \usepackage[T1]{fontenc}
\usepackage[french]{babel}
\usepackage{comment}
\usepackage{subcaption}
\usepackage{subfiles}
\usepackage{graphicx}
% \usepackage{minted}
\usepackage{placeins}
\graphicspath{ {./img/} }



\title{Panic At Tortuga \thanks{From LifeInvaders Production Team}}

\author{
    KEDJNAN
    E, Harrys\\
    \texttt{harrys.kedjnane@epita.fr}
    \and
    DEVERS, Renaud-Dov\\
    \texttt{renaud-dov.devers@epita.fr}
    \and
    COHEN-SCALI, Julien\\
    \texttt{julien.cohen-scali@epita.fr}
    \and
    DE LA PORTE DES VAUX, Paul\\
    \texttt{paul.de-la-porte-des-vaux@epita.fr}
    
}
% \date{February 2014}

\begin{document}

\begin{titlepage}
\maketitle
\end{titlepage}

\begin{center}
    \textbf{Introduction}

    Nous allons vous présenter notre projet de S2.
    Notre équipe a choisi de réaliser un jeu vidéo pour le projet de l'Info SUP.
\end{center}

\tableofcontents
\newpage

\section{Introduction}
\subsection{Présentation du jeu}
\begin{flushleft}
    Panic At Tortuga est un jeu multijoueur dans lequel les différents joueurs doivent s'entre-éliminer sur une île.
    Le jeu s'inspire de jeux connus : Le premier est le mod "Guess Who ?"\footnote{Lui même inspiré de jeux tels que Hide \& Seek, Spy Party ou encore Prop Hunt} de Garry's Mod
        où une première équipe doit rechercher et éliminer l'autre équipe qui s'est déguisée en NPC (Non-Player Character) et qui doit imiter les mouvements aléatoires du bot.
    Le deuxième jeu dont nous nous sommes inspirés est le mode multijoueur d'Assasin's Creed Brotherhood\footnote{Jeu sorti en 2010}.
    Dans ce mode, chacun des joueurs se voit attribuer une cible, ce qui fait que tous les joueurs sont à la fois des traqueurs et traqués.

    Notre jeu se passera sur une île tropicale fictive dans les Caraïbes.
    L'objectif est de se cacher parmi la foule d'une petite île perdue au beau milieu de l'océan et d'éliminer votre cible sans être vu.
\end{flushleft}

\subsection{L'équipe de LifeInvaders}
\textbf{DEVERS Renaud-Dov (Chef de Projet)} \newline
J'adore l'informatique depuis des années,
je suis tombé dans la marmitte des vieux ordinateurs (Windows 2000 !) à disquette quand j'était petit.

J'aime les travaux et les projets de groupe, en ayant 3 à mon actif, dont le projet d'ISN et celui de SI. J'ai pu réaliser par exemple une application dynamique pour libraires avec une connexion à la database de Google Books.
J'y ai appris le rôle d'un chef de groupe, de l'organisation à la responsabilité, je trouve les projets formateurs.



\textbf{KEDJNANE Harrys} \newline


\textbf{DE LA PORTE DES VAUX Paul} \newline

\textbf{COHEN-SCALI Julien} \newline

\section{Choix techniques}

Nous avons décidé de réaliser un jeu en 3D sur le moteur cross-platform de Unity Technologies.
Ce moteur est complet et possède plein de fonctionalités pour créer un environnement jeu fonctionnel.

Nous avons décidé d'acheter un asset contenant de multiples textures, prefabs, objets et personnages sur le thème des pirates et des Caraïbes.\footnote{Voir Partie Coûts de production}
Nous n'aurons donc pas, sauf quelques rares modèles, besoin de modéliser des mesh sur Blender.

Nous avons choisi d'intégrer l'outil PUN 2 de Photon pour la réalisation du multijoueur et l'interraction entre joueurs.


% \newpage
\section{Répartition des tâches}
\begin{center}
    \begin{tabular}{|c||c|}
        \hline
        \textbf{Tâche} & \textbf{Nom} \tabularnewline
        \hline
        Création de la carte & Paul\tabularnewline
        \hline
        Réalisation du controle du personnage (+Animations) & Renaud-Dov\tabularnewline
        \hline
        Implémentation des mécaniques du jeu (game core) & Renaud-Dov et Harrys\tabularnewline
        \hline
        Réalisation de l'IA & Renaud-Dov et Paul \tabularnewline
        \hline
        Implémentation du multijoueur & Julien et Harrys \tabularnewline
        \hline
    \end{tabular}
\end{center}

% \newpage
\subfile{sections/mouvements.tex}
% \newpage
\subfile{sections/ai.tex}
% \newpage
\subfile{sections/multiplayer.tex}
% \newpage
% \subfile{sections/costs.tex}
\section{Autres liens utiles}

\end{document}