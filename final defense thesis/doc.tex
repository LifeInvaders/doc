\documentclass[12pt, french]{article}

\usepackage{geometry}
 \geometry{
 a4paper,
 total={150mm,240mm},
 left=30mm,
 top=30mm,
 }


\usepackage{helvet}

\usepackage[utf8]{inputenc}
\usepackage[T1]{fontenc}
\usepackage[french]{babel}
\usepackage{comment}
\usepackage{subcaption}
\usepackage{subfiles}
\usepackage{graphicx}
\usepackage{diagbox}
\usepackage[table,xcdraw]{xcolor}
% \usepackage{minted}
\usepackage{placeins}

\usepackage{fancyhdr}



\graphicspath{ {./img/} }

\title{\fontfamily{phv}\selectfont \Huge \textbf{Panic At Tortuga}}
\author{\fontfamily{phv}\Huge{Rapport de projet}}
\date{\fontfamily{phv}\selectfont Juin 2021}

\begin{document}

\begin{titlepage}
    \maketitle
    
    \thispagestyle{empty}
    % {\fontencoding{T1}\fontfamily{calligra}\selectfont the font is temporarily changed}
    \vspace{10pt}
    \begin{figure}[hbt!]
        \centering
        \includegraphics[scale=0.4]{logo.png}
    \end{figure}
    \vspace{10pt}

    \begin{figure}[hbt!]
        \centering
        \includegraphics[scale=0.8]{logo_lifeinvaders_copie.png}
    \end{figure}
\end{titlepage}

% ////////////////////////////////////////////////
% ////////////////////////////////////////////////
% ////////////////////////////////////////////////


\tableofcontents
\vspace{2cm}

\pagestyle{fancy}
\lhead{Panic At Tortuga}
\fancyhead[C]{Avril 2021}
\rhead{LifeInvaders Production}

\section{Introduction}

L'heure du rendu final a sonné. Cette dernière période de travail n'a pas été la plus calme, car rythmée par un TD de maths et les partiels de fin d'année.
Cependant, le projet n'a pas cessé d'évoluer pour atteindre sa forme finale, plutôt aboutie. Cette dernière semaine en particulier a été une semaine 
de travail intense, à travailler de 9h à 2h pour peaufiner le jeu dans ses moindres détails et produire le rapport le plus complet possible.
Ce rapport est organisée en parties du développement, elles-mêmes divisées en périodes, afin de pouvoir facilement suivre l'avancement d'un 
aspect du projet sur les trois périodes de développement correspondant aux trois soutenances. Chaque sous-partie présente sous forme de paragraphes 
les travaux effectués lors de la période en question.

\section{Présentations et impressions}
\subfile{sections/presentation.tex}
\subfile{sections/impressions.tex}
\subfile{sections/individual_work.tex}
\section{Développement}
\subfile{sections/timeline.tex}
\subfile{sections/progression.tex}
\subfile{sections/launcher.tex}
\subfile{sections/ia.tex}
\subfile{sections/map.tex}
\subfile{sections/ui.tex}
\subfile{sections/website.tex}
\subfile{sections/multiplayer.tex}
\subfile{sections/gameplay.tex}
\subfile{sections/graphismes.tex}
\subfile{sections/prev.tex}
\subfile{sections/bibliographie.tex}
\newpage

\section{Conclusion}

Le projet Panic at Tortuga aura été un réel défi technique, et nous aura demandé beaucoup d'investissement pour arriver à sa forme finale. Tous les éléments, 
des mécaniques de jeu au multijoueur, auront représenté des centaines d'heures de travail, pendant lesquelles nous avons gagné en compétences et appris à utiliser 
Unity, ce que nous serons peut-être amenés à refaire dans notre vie professionelle. Certaines tâches, comme la carte, nous auront enseigné la patience, et d'autres 
comme la synchronisation Photon, nous auront appris à nous dépasser. Les rapports de soutenance, par la régularité qu'ils imposent, ont permis un avancement 
constant du jeu, sans périodes d'inactivité prolongée. 

\section{Remerciements}

Nous tenons à remercier :


\paragraph{M. Ternier}qui a pris le temps de s'intéresser 
à notre projet, en lisant nos rapports et en nous posant des questions pertinentes lors des soutenances, signe 
de son intérêt. 

\paragraph{Mme. Devers}qui nous a fait des tacos maison et du risotto aux asperges 
pendant la semaine de rush, quand nous n'avions pas le temps de cuisiner.

\paragraph{Pablo Mira}qui nous a bien fait rire pendant nos quarts d'heure de pause.



\listoffigures
\listoftables
\end{document}
