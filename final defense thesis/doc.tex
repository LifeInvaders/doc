\documentclass[french, 15pt]{article}

\usepackage{geometry}
 \geometry{
 a4paper,
 total={150mm,240mm},
 left=30mm,
 top=30mm,
 }


\usepackage{helvet}

\usepackage[utf8]{inputenc}
\usepackage[T1]{fontenc}
\usepackage[french]{babel}
\usepackage{comment}
\usepackage{subcaption}
\usepackage{subfiles}
\usepackage{graphicx}
\usepackage{diagbox}
\usepackage[table,xcdraw]{xcolor}
% \usepackage{minted}
\usepackage{placeins}

\usepackage{fancyhdr}

\setcounter{secnumdepth}{4}

\graphicspath{ {./img/} }

\title{\fontfamily{phv}\selectfont \Huge \textbf{Panic At Tortuga}}
\author{\fontfamily{phv}\Huge{Rapport de projet}}
\date{\fontfamily{phv}\selectfont Juin 2021}

\begin{document}

\begin{titlepage}
    \maketitle
    
    \thispagestyle{empty}
    % {\fontencoding{T1}\fontfamily{calligra}\selectfont the font is temporarily changed}
    \vspace{10pt}
    \begin{figure}[hbt!]
        \centering
        \includegraphics[scale=0.43]{logo.png}
    \end{figure}
    \vspace{10pt}

    \begin{figure}[hbt!]
        \centering
        \includegraphics[scale=0.84]{logo_lifeinvaders_copie.png}
    \end{figure}
\end{titlepage}

% ////////////////////////////////////////////////
% ////////////////////////////////////////////////
% ////////////////////////////////////////////////


\tableofcontents
\newpage

\pagestyle{fancy}
\lhead{Panic At Tortuga}
\fancyhead[C]{Avril 2021}
\rhead{LifeInvaders Production}

\section{Introduction}

% introduction à compléter
L'heure du rendu final a sonné. Cette dernière période de travail n'a pas été la plus calme, car rythmée par un TD de maths et les partiels de fin d'année.
Cependant, le projet n'a pas cessé d'évoluer pour atteindre sa forme finale, plutôt aboutie.

\newpage
\section{Développement}
\subfile{sections/timeline.tex}
\subfile{sections/progression.tex}
\subfile{sections/launcher.tex}
\subfile{sections/ia.tex}
\subfile{sections/map.tex}
\subfile{sections/website.tex}
\subfile{sections/multiplayer.tex}
\subfile{sections/gameplay.tex}
\subfile{sections/graphismes.tex}
\subfile{sections/prev.tex}
\newpage
\section{Conclusion}
% conclusion à changer
En définitive, le projet est sur une bonne lancée en dépit de quelques contre-temps.
L’implémentation du multijoueur progresse à grands pas, de la même façon que les éléments
client-side (animations, mouvements, shaders…) et que la carte. Le site internet permet
maintenant de suivre le projet mois par mois, et le launcher affiche les nouveautés et permet de rester à jour.
\listoffigures
\listoftables
\end{document}
