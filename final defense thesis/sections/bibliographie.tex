\subsection{Bibliographie et références}

    \subsubsection{Logiciels utilisés}

    
        \paragraph{Github} Nous avons utilisé GitHub pour centraliser nos sources, et utilisé les outils Pages et CI respectivement 
        pour l'hébergement du site web et pour notre usine à logiciel.

        \paragraph{Electron} Cela nous a semblé une option fiable, et déployable facilement sur de multiples plateformes. De plus, 
        un framework beaucoup utilisé, ce qui nous permet d'obtenir des réponses à nos questions rapidement grâce à une communauté importante.

        \paragraph{Inno Setup} Ce logiciel permet de créer assez rapidement un installer / uninstaller fonctionnel ; bien que très classique 
        (pas d'options pour le personnaliser), il est efficace et bien documenté.

        \paragraph{LaTeX Workshop} Cette extension pour Visual Studio Code permet d'écrire et de compiler les fichiers LaTeX (cahier des charges, 
        rapports de soutenance, manuels).

    \subsubsection{Assets Unity}


        \paragraph{Polygon Pirate Pack} Cet asset nous a permis de nous concentrer plus sur le rendu final du jeu (notamment 
        les animations et les mécaniques) que sur la modélisation 3D, qui est longue et rébarbative. Cet asset nous a coûté 19,99€ 
        et nous a fait économiser de nombreuses heures de recherche d'assets sur Unity Store et de modélisation sur Blender. 
        Nous tenons à rappeler que c'est l'unique dépense du jeu, et que tout le reste de la partie graphique (effets, shaders,  
        post-processing, conception de la carte...).

        \paragraph{Mixamo} Librairie d'animations en ligne, Mixamo est un référence en matière d'animations de personnages. La grande 
        majorité des animations du jeu proviennent de Mixamo, et certaines ont été remodifiées.

        \paragraph{Real-time procedural cable} Asset permettant de générer des câbles dans le jeu, statiques ou dynamiques et soumis 
        à la gravité ou non. Cet asset nous a servi à faire les tyroliennes de façon plus simple qu'avec un parallélépipède allongé. 

        
    \subsubsection{Médias}

        Les médias, audio ou vidéo, utilisés dans le jeu ou dans toute autre production du groupe, sont les suivants :

        \begin{itemize}
            \item Musique du trailer : Blackbeard - Derek Fietcher
            \item Musique du jeu : The buccaneer's Haul - Silverman Sound Studios
            \item Ecran de chargement : Malchev @ Getty Images
            \item Animations : Banque de données Mixamo
            \item Sons : bibliothèque ZapSplat
        \end{itemize}