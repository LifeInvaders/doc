\subsection{L'équipe de LifeInvaders}
\vspace{0.5cm}

    \paragraph{DEVERS Renaud-Dov (Chef de Projet)}
    J'adore l'informatique depuis des années, je suis tombé dans la marmite des vieux ordinateurs (Windows 2000 !) à disquette quand j'était petit.
    J'aime les travaux et les projets de groupe, en ayant trois à mon actif, dont le projet d'ISN et celui de SI. J'ai pu réaliser par exemple une application dynamique pour libraires avec une connexion à la database de Google Books.
    J'y ai appris le rôle d'un chef de groupe, de l'organisation à la responsabilité, je trouve les projets formateurs.
    Ce projet sera donc l'occasion de m'améliorer dans le travail de groupe, pour être prêt pour le monde professionnel.
    \vspace{0.5cm}


    \paragraph{KEDJNANE Harrys}
    Avant EPITA, je faisais une PACES et je n’avais jamais codé même si l’informatique est un sujet qui m’a toujours fortement intéressé. Depuis mon arrivé à cette école, j’ai découvert les joies de la programmation et ce projet me permettra d’approfondir ce sujet appliqué à un autre milieu qui m’intéresse tout autant: le jeu vidéo. 
    \vspace{0.5cm}

    \paragraph{DE LA PORTE DES VAUX Paul}
    Ayant découvert le monde de l’informatique assez tard, je n’ai fait que des petits projets, comme des Hackintosh ou des projets python de cryptographie et de reconnaissance d’objets.
    La création d’un jeu en 3D m’enchante donc, car elle va me permettre d’explorer de nouveaux horizons.
    \vspace{0.5cm}

    \paragraph{COHEN-SCALI Julien}
    Étant un gros joueur et passionné depuis mon enfance, ce projet est un moyen pour moi de m’exprimer grâce aux différents cours de programmation en C\#.
    Par ailleurs, ayant fait un stage à l’ISART, le C\# et Unity ne me sont pas inconnus. De plus, ce projet est un bon moyen de s’exercer à la Programmation Orientée Objet vue lors derniers cours de programmation,
    ce qui permettra à notre groupe d’améliorer notre code et d’acquérir une méthode bien plus professionnelle, nous préparant à notre vie future.
    La collaboration autour du projet et l’importance de travailler en groupe grâce a git développe en chacun de nous un esprit d’équipe conséquent.
    Ce projet est donc, pour moi, un très bon entrainement pour le développement de projet dans ma vie futur. (mais aussi un très bon moyen de s’amuser).
    \vspace{0.5cm}