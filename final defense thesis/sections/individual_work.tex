\subsection{Détail du travail effectué par chaque membre de l'équipe}

    En vue de la notation individuelle de la troisième soutenance, il nous a semblé important de rendre à César ce qui est à César, en identifiant 
    clairement le travail fourni par chaque membre du groupe. il s'agit de quelque chose que nous n'avons volontairement pas fait dans les autres soutenances, 
    du fait de la notation collective et par respect pour ceux qui n'auraient, pour une raison quelconque, pas pu travailler autant qu'ils l'auraient dû sur un période. 
    Le travail nous a semblé assez équilibré, en nous n'avons pas eu l'impression de traîner des membres de l'équipe, car chacun y a mis du sien pour faire 
    avancer le projet.


    \subsubsection{Harrys Kedjnane}

        Harrys s'est occupé de presque toute la partie multijoueur, de l'instanciation Photon à la synchronisation des cartes. Son travail 
        consistait donc principalement à analyser ce qui était fait en local, et à le porter en multijoueur.

    \subsubsection{Julien Cohen-Sclai}

        Julien a eu la gentillesse de continuer de travailler pour le groupe, alors même qu'il change d'orientation l'an prochain pour aller 
        faire le Bachelor sécurité d'EPITA à la Défense. Il s'est principalement occupé d'aider Harrys au début sur la partie multijoueur, 
        puis a fait le frontend du HUD, le menu multijoueur et le tableau des scores.

    \subsubsection{Renaud-Dov Devers}

        Dov a fait quasiment toutes les mécaniques de jeu : du player controller aux échelles, il s'est occupé de l'input system, des scripts 
        pour les déplacements et des innombrables résolutions de bugs liés aux nombreuses possibilités. Ainsi, il a dû corriger tous les problèmes d'animations 
        (par exemple, si le personnage court et prend une échelle, il faut d'abord désactiver la course du joueur, puis activer l'échelle).

    \subsubsection{Paul de La Porte des Vaux}

        Paul s'est occupé de créer les cartes du jeu et de peaufiner leurs détails, ainsi que les navmeshes et les tyroliennes. Il a également 
        rédigé les rapports en LaTeX, les manuels d'utilisation et le site internet.