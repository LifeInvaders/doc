\subsection{Site Internet}

    \subsubsection{Première soutenance}

        Au moment de la première soutenance, seule une page HTML de démo était disponible sur le site, afin de vérifier 
        que la mise en place de GitHub Pages avait bien marché. Un template HTML, plus abouti, était également hébergé en local, 
        mais nous avons finalement décidé d'utiliser GitHub Pages, premièrement pour son accessibilité rendant le site modifiable 
        plus rapidement par n'importe quel membre de l'équipe, et aussi pour des raisons évidentes de sécurité et de disponibilité 
        (pas terrible d'ouvrir un port de sa box, et je n'allais pas ouvrir un port à chaque fois lors de mes déplacements).

    \subsubsection{Deuxième soutenance}

        \paragraph{réalisation}

            La réalisation d'un site internet pour le projet était un objectif
            programmé pour la deuxième soutenance, ce qui est maintenant chose faite.
            Pour ce dernier, nous avons retenu Bootstrap, qui est une collection
            d'outils HTML, CSS et Javascript apportant des éléments de site
            esthétiques et simples à uttimelineependant, comme c'est un thème très 
            communément utilisé, nous changerons à terme certains éléments afin 
            d'y apposer notre signature. Bootstrap Studio nous a aidés à réaliser
            un site dynamique, mais ne s'est pas montré à la hauteur de nos espérances
            en matière de personnalisation. En effet, on ne peut pas modifier
            le code HTML (seulement ajouter / supprimer des blocs ou éditer les
            attributs), et les styles CSS ne sont modifiables qu'en créant une copie
            du fichier CSS d'origine.

            \begin{figure}[hbt!]
                \centering
                \includegraphics[scale=0.365]{bootstrap_studio.png}
                \caption{Le logiciel Bootstrap Studio}
            \end{figure}

            La conception du site s'est donc déroulée en deux parties. En premier, la création
            d'une base grâce à BootStrap Studio, en modifiant les images, textes et titres présents, ainsi
            que d'autres chose un peu plus minutieuses, comme la modification de tableaux et des encarts personnalisés.
            Ensuite, la modification plus poussée des options verouillées par BootStrap Studio, comme l'ajout de liens 
            sur les images, ou encore la modification des titres et favicons.

            \begin{figure}[hbt!]
                \centering
                \includegraphics[scale=0.365]{website.png}
                \caption{Aperçu de notre site}
            \end{figure}
            \FloatBarrier


        \paragraph{Hébergement}

            L'hebergement de petits sites comme celui-ci n'étant pas très contraignant,
            nous avons décidé d'utiliser un hébergeur gratuit, car ces derniers
            sont généralement largement suffisants. Après avoir cherché parmi les
            solutions proposées, nous avons décidé d'utiliser la solution 
            \textit{Github Pages}, qui permettait d'avoir une extension "sérieuse"
            (nous préférons une site qui finit par github.io que par wix.com),
            ainsi qu'une gestion de ce dernier très simplifiée, grâce au gestionnaire de versions.
            Ainsi, tout comme pour le projet, les versions sont gérées en trois commandes 
            (git add, git commit , git push), et la limite de taille de 1 Go est plus que suffisante pour quatre pages html. 
            

        
        \paragraph{Améliorations futures}

            La création et la maintenance d'un page recensant les nouveautés apportées par chaque version est un des 
            principaux points prévus pour la dernière soutenance, afin de pouvoir voir l'évolution du projet au fil des 
            mois et d'être informé des dernières mises à jour du jeu. Un autre aspect important à développer est l'indentité
            visuelle du site, afin de le rendre unique. Pour cela,  nous comptons utiliser de polices de caractères originales 
            et de couleurs vives et attrayantes. Des éléments dynamiques seront également nécéssaires à rendre le site agréable 
            à consulter.

    \subsubsection{Dernière soutenance}

        Maintenant que le site est fonctionnel, la dernière période consiste simplement à le maintenir à jour. Ainsi, deux cases 
        ont été ajoutées à la chronologie et des liens ont été mis à jour. Nous avons également rajouté des boutons de téléchargement 
        des rapports de soutenance, et amélioré la description du jeu.