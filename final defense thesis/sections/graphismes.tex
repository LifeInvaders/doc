\subsection{Graphismes}

    \vspace{0.5cm}
    \subsubsection{Première soutenance}
    \vspace{0.5cm}

        Nous avons fait le choix d'acheter un Asset low-poly proposant un univers pirate, aussi esthétique que léger. Les menus seront 
        designés avec un effet parchemin (bordure irrégulière) pour évoquer l'univers pirate et les polices seront stylisées pour la même raison. 
        

    \vspace{0.5cm}
    \subsubsection{Deuxième soutenance}
    \vspace{0.5cm}

        Pour rendre le jeu plus beau, nous avons intégré plusieurs effets.
        Afin de les créer, nous avons utilisé plusieurs outils de Unity :
        Shader Graph, TimeLine, Visual Effect FX Graph, Cinemachine.

        Voici plusieurs exemples des éléments graphiques que nous avons pu réaliser : 

        \paragraph{Shader Eau}

        L'eau que nous avions était plate mais surtout n'était pas animée.
        Plutôt que d'acheter un asset permettant d'avoir de l'eau animée,
        nous avons opté pour une réalisation manuelle à l'aide de l'outil Shader Graph.

        Il y a trois blocs pour les vagues (les petites, les moyennes et les grandes),
        ainsi que des blocs sinusoïdaux pour que chaque bloc d'eau soit aligné et synchronisé avec les autres.
        \begin{figure}[hbt!]
            \centering
            \includegraphics[scale=0.2]{waterShaderGraph.png}
            \caption{Shader Graph de l'eau}
        \end{figure}
        
        
        \begin{figure}[hbt!]
            \begin{subfigure}[b]{0.49\textwidth}
                \includegraphics[scale=0.17]{waterBefore.png}
            \subcaption{Shader de l'eau par défaut}
            \end{subfigure}
            \begin{subfigure}[b]{0.49\textwidth}
                \includegraphics[scale=0.17]{WaterAfter.png}
                \subcaption{L'eau avec son shader personnalisé}
            \end{subfigure}
            \caption{Amélioration de l'eau}
        \end{figure}
        \FloatBarrier



        \paragraph{Mode Jour/Nuit}
        Le mode jour/nuit permet, au lancement de la partie,
        de choisir soit le jour soit la nuit. Si l'on choisit le mode nuit, un script désactive la lampe de jour,
        pour la remplacer par une autre plus sombre. Des effets de pluie ont été rajoutés en plus d'effets de post processing.
        Mais l'effet le plus important est d'activer toutes les lampes de la carte.
    

        \begin{figure}[hbt!]
            \begin{subfigure}[b]{0.49\textwidth}
                \includegraphics[scale=0.14]{day.png}
            \end{subfigure}
            \begin{subfigure}[b]{0.5\textwidth}
                \includegraphics[scale=0.14]{night.png}
            \end{subfigure}
            \caption{L'église de jour et de nuit}
        \end{figure}
        \FloatBarrier

        \paragraph{Shader de disparition/apparition}
        Pour éviter de faire apparaître ou faire disparaitre en un instant les personnages, nous avons décidé de créer un shader qui qui selon un matériau en paramètre, 
        fait une transition de l'état initial vers l'état final. 
        \begin{figure}[hbt!]
            \centering
            \includegraphics[scale=0.5]{dissolve.png}
            \caption{Application du shader sur un personnage}

        \end{figure}
        \FloatBarrier

        Nous avons rencontré un problème lorsque nous appliquions le matériau avec le shader,
        les paramètres de l'instance du shader sur le matériau étant communs. Ainsi, lorsqu'un premier personnage mourrait juste après un autre, 
        le premier mort recommençait l'animation de transition en plus de changer de couleur de vêtements et de peau.
        La solution trouvée, toute simple, consiste à créer un nouveau matériau et d'appliquer  ses paramètres dans le script.


        \paragraph{Radar}

        Le radar fonctionne de manière très simple : une fois la cible attribuée au joueur, la boussole va pointer vers cette dernière, de plus 
        en plus approximativement lorsque le joueur s'en rapproche. Elle est affichée en bas à droite de l'écran, pour ne pas réduire la 
        visibilité, et est le meilleur ami du joueur au cours de sa traque, car elle s'avère très précise à longue distance.
        D'un point de vue technique, la variable \textit{distance} du shader est mise à jour constamment, et un shader modifié en conséquence 
        est appliqué à une sphère affichée en bas de l'écran, ce qui paraît assez peu intuitif comme méthode mais donne un rendu très correct. 
        

        \paragraph{Post Processing}

        Nous avons ajouté le package Post Processing de Unity qui avec l'URP (Universal Render Pipeline)
        nous a permis de créer des ambiances et effets visuels agréables.
        Nous avons créé des volumes qui activent les effets de caméra.
        Cela nous a permis notamment d'avoir un effet vieux film/sépia lorsque le joueur passe en mode verrouillage.

        \begin{figure}[hbt!]
            \centering
            \includegraphics[scale=0.2]{sepia_mode.png}
            \caption{Mode verrouillage}
        \end{figure}
        \FloatBarrier


        Si l'on s'attarde sur cette figure, nous voyons que l'effet "sépia" n'est pas visible sur certains objets.
        C'est là toute la difficulté d'avoir des effets de Post Processing différents sur chaque couche.
        Nous avons donc ajouté plusieurs caméras \textit{Overlay} superposées sur la caméra principale qui affichent chacune une partie de l'image.
        Après avoir réglé des soucis de profondeur sur les caméras, nous avons obtenu l'effet précédent.

    \vspace{0.5cm}
    \subsubsection{Dernière soutenance}
    \vspace{0.5cm}

        \paragraph{Trailer du projet}

        Pour la soutenance finale, nous avons réalisé un trailer d'une minute montrant les deux cartes. Cette vidéo nous a pris 
        20 heures à réaliser sous Première Pro, car nous avons dû réaliser de nombreuses prises et avons utilisé des calques 
        pour les transitions. Ces transitions ont été réalisé image par image sur une vidéo à 60 images/s, en modifiant légèrement 
        le calque à la main. Le rendu final montre donc des animations très travaillées et abouties.