\subsection{Interface utilisateur}
    
    \vspace{0.5cm}
    \subsubsection{Cahier des charges}
    \vspace{0.5cm}

        Afin de faire l’UI, nous avons majoritairement utilisé l’outil canvas de Unity. 
        Pour la direction artistique, compte tenu du thème de notre jeu portant sur les pirates, 
        nous avons décidé de nous inspirer du menu principal du jeu Sea Of Thieves.
        

        \begin{figure}[hbt!]
            \centering
            \includegraphics[scale=0.22]{img/sotmenu.png}
            \caption{Menu principal du jeu Sea of Thieves}
        \end{figure}
        \begin{figure}[hbt!]
            \centering
            \includegraphics[scale=0.3]{img/menu.png}
            \caption{Préversion de notre menu}
        \end{figure}
        \FloatBarrier
        
        Afin de rendre le menu plus vivant, 
        nous pourrions utiliser des animations sur les boutons ainsi que sur l’arrière-plan du menu. 
        Notre menu intègrera 4 boutons et un champ. Le champ permettra au joueur d’entrer son nom dans le jeu.
        Les 4 autres boutons serviront à :
        \begin{itemize}
                \item Se connecter à une partie
                \item Quitter le jeu
                \item Régler les paramètres du jeu
                \item Personnaliser son personnage
        \end{itemize}


    \vspace{0.5cm}
    \subsubsection{Première soutenance}
    \vspace{0.5cm}

        Les interfaces sont une partie importante de tout jeu. Elles permettent de donner au joueur plus de contrôle sur son expérience 
        et de présenter les informations nécessaires de façon claire, facilement accessible et si possible esthétique. Un progrès 
        important a été fait quant à la création d'interfaces.
        
        \paragraph{Tableau des scores}

            Pendant le déroulement du jeu, le joueur a besoin de connaitre certaines informations concernant son personnage, la partie et les autres joueurs. 
            Pour fournir ces informations, nous avons opté pour un tableau des scores disponible à n’importe quel moment de la partie. Ce tableau permet également un classement simple, clair et rapide des joueurs en fonction de leurs points.
            La difficulté principale de ce tableau des scores est la synchronisation entre un script coté client qui doit actualiser à chaque pression de la touche TAB le score ainsi que la présence de chaque joueur et la récupération des données de chaque Player, mises à jour en temps réel sur le serveur.

        \paragraph{HUD}

            Le HUD est une interface présente tout le temps à l'écran, et permettant d'afficher 
            les informations relatives au joueur et à l'avancement de la partie directement en jeu. 
            Ce dernier a donc pour obligation d'être complet (au moins : informations sur la cible,
            points, temps de jeu restant) et assez discret, pour ne pas avoir l'impression 
            de jouer à travers un hublot. C'est pourquoi nous avons décidé de mettre les 
            éléments dans les coins le plus possible, afin de ne pas déranger les joueurs.
            Les informations fournies par le HUD n'étant pas utiles au stade actuel du jeu 
            (encore assez expérimental), Paul a décidé de se focaliser sur la carte et sur la manipulation
            générale des scripts Unity avant de le commencer, et ce dernier n'est donc pas 
            encore implémenté dans les builds de la première soutenance.


        \paragraph{Menu Principal}

            Le menu d'accueil est la première chose que le joueur voit en lançant le jeu.
            Celui-ci est actuellement entièrement implémenté et fonctionnel. Cependant il continuera 
            d'évoluer au fil du projet et de ses besoins.
            Le thème du menu principal doit être représentatif du jeu. Il doit donc bien entendu 
            afficher la couleur dès le lancement: ceci est un jeu de pirates.
            Le fond, les couleurs, la musique... Y est également présent le logo du 
            groupe ainsi que celui du jeu.
            Plusieurs options sont disponibles. Le joueur peut tout d'abord se connecter 
            au serveur Photon. Il accède ainsi à un menu lui permettant de rejoindre une 
            salle. Il peut également y modifier son pseudo, qui sera affiché aux autres 
            joueurs. Seule la fonction "Quick Play" a été implémentée (cf. multijoueur) mais 
            l'objectif à long terme est d'ajouter une liste ainsi qu'un menu de création de 
            salles. Cette partie du menu principal a été réalisée par Julien.
            Il est également possible de modifier les paramètres de jeu comme la 
            résolution, le mode plein écran, le volume, ou encore remapper les touches de 
            contrôles. Ces changements sont sauvegardés (cf. système de sauvegardes).
            Un menu de personnalisation de personnage a également été implémenté 
            (cf. système de progression). On peut y modifier certaines caractéristiques, 
            comme la couleur des vêtements ou de la peau. Un aperçu du modèle est disponible 
            (et peut pivoter grâce à la souris). Les paramètres, ainsi que ce menu, ont été 
            réalisés par Harrys.
            Enfin, on retrouve (évidemment) un bouton quitter pour fermer l'application.
            
            \begin{figure}[hbt!]
                \centering
                \captionsetup{justification=centering}
                \begin{subfigure}[b]{0.3\textwidth}
                    \includegraphics[scale=0.1]{mainmenu.png} 
                    \caption{Menu principal}
                \end{subfigure}
                \hspace{150pt}
                \begin{subfigure}[b]{0.3\textwidth}
                    \includegraphics[scale=0.1]{settingsmenu.jpg} 
                    \caption{Paramètres}
                \end{subfigure}
                \begin{subfigure}[b]{0.3\textwidth}
                    \includegraphics[scale=0.1]{charcustom.jpg} 
                    \caption{Apparence}
                \end{subfigure}
                \begin{subfigure}[b]{0.3\textwidth}
                    \includegraphics[scale=0.1]{photonmenu.png} 
                    \caption{Menu multijoueur}
                \end{subfigure}
                \caption{Différents affichages du menu d'accueil}
            \end{figure}


    \vspace{0.5cm}
    \subsubsection{Deuxième soutenance}
    \vspace{0.5cm}
    
        \paragraph{Menu principal}
        
            Le menu principal a été amélioré, afin d'y ajouter des paramètres comme la sélection du mode de fenêtre (fenêtré 
            ou plein-écran) ou le format de la fenêtre (compatible avec des formats 16:9, 4:3 et 16:10).
            Il comporte maintenant les quatre options annoncées dans le cahier des charges, à savoir jouer, modifier le personnage, 
            régler les paramètres et quitter.

            Un début de boussole a aussi implémenté, mais n'est pas accessible en jeu car encore au stade de développement. Cette dernière, 
            présente en bas à droite de l'écran, permet de localiser sa cible (avec une précision proportionnelle à la distance à la cible).


    \vspace{0.5cm}
    \subsubsection{Troisième soutenance}
    \vspace{0.5cm}
    
        \paragraph{Menu Principal}
        
            Le menu principal a encore été amélioré, et est maintenant très semblable à celui de Sea of Thieves, avec des boutons recouverts 
            par des images pour donner du cachet au jeu. Des danses de victoire ayant également été ajoutées au jeu, un menu de sélection est maintenant 
            disponible dans les options de configuration du joueur, au même endroit que le choix de l'apparence.
        
            \begin{figure}[hbt!]
                \centering
                \includegraphics[scale=0.4]{img/mainmenu.png}
                \caption{Version antérieure du menu principal}
            \end{figure}

            \begin{figure}[hbt!]
                \centering
                \includegraphics[scale=0.36]{img/menu_principal.png}
                \caption{Dernière version du menu}
            \end{figure}
            \FloatBarrier
        
        \paragraph{Liste des joueurs et des scores}

            Le scoreboard, développé par Julien, permet de connaitre en temps réel les scores, pings et pseudos des autres joueurs de la partie. 
            Il s'appuie sur une liste de joueurs qui est mise à jour à chaque assassinat puis transmise aux joueurs qui ouvrent cette liste. 
            La difficulté du scoreboard était de rajouter un système pour trier les joueurs par score décroissant, car cela nécéssite de 
            détruire tous les enfants de la liste (qui correspondent à ses  entrées), et les rajouter dans l'ordre à chaque fois.

        \paragraph{Menu de sélection de partie}

            Le menu de sélection de partie permet de choisir de rejoindre une room publique ou privée. Pour rejoindre une room publique, le jeu 
            connecte simplement le joueur à n'importe quelle room ouverte. En revanche, le système de parties privées est bien plus complexe :
            pour créer une partie privée, le jeu génère une clé (5 chiffres), qui correspond au nom de la room, et masque cette room afin qu'elle 
            ne soit pas joignable par les joueurs publiques. Ensuite, les personnes qui veulent rejoindre cette partie doivent entrer la clé pour 
            indiquer au serveur quelle room ils veulent rejoindre. En définitive, les parties privées sont donc de simples parties cachées des autres 
            joueurs et accessibles avec un code.

        