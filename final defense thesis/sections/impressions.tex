\subsection{Impressions des membres de l'équipe}

    \vspace{0.5cm}
    \subsubsection{Développement du jeu}
    \vspace{0.5cm}

        Le développement de ce jeu s'est avéré être un vrai défi technique, avec de nouvelles compétences à apprendre. 
        Le logiciel Unity est très riche en fonctionnalités, et de ce fait assez complexe à utiliser de par son interface 
        recouverte d'options. Cependant, passée cette appréhension, cet éditeur se montre pratique pour certaines tâches répétitives, et 
        on s'habitue vite à son utilisation. Nous avons développé les scripts en C\# sur Jetbrains Rider, qui est un outil 
        excellent, d'une simplicité d'utilisation remarquable et incluant des outils très utiles comme git. 


    \vspace{0.5cm}
    \subsubsection{Gestion du planning et des tâches}
    \vspace{0.5cm}

        Le planning était assez difficile à tenir, car nous avions prévu d'avoir fini 80\% du jeu à la deuxième soutenance. 
        Cependant, il s'agissait d'un choix du groupe, et cela nous a permis de ne pas nous laisser submerger par le travail 
        à finir. Le système de progression et le HUD sont les deux aspects du jeu où nous avons pris un peu de retard, car ils nous ont 
        semblé moins importants que d'autres comme le multijoueur ou le gameplay. Afin de mieux tenir le planning, nous avons mis en place 
        un système d'issues permettant de trier les tâches restantes par priorité ou par importance, et de les catégoriser ou les fermer à l'aide 
        d'un kanban. 


    \vspace{0.5cm}
    \subsubsection{Difficultés rencontrées}
    \vspace{0.5cm}

        Les difficultés rencontrées ont été nombreuses : au début, elles étaient surtout liées à l'utilisation de Git, car ayant moins d'éléments dans notre jeu, 
        nous modifiions souvent les mêmes fichiers, et avions donc des \textit{Merge conflicts} difficiles et fastidieux à résoudre. 
        Ensuite, nous perdions systématiquement nous configurations, car elles n'étaient pas ajoutées au fichier gitignore. 
        Une autre difficulté a été l'implémentation du multijoueur, car la documentation Photon, bien que très complète, est longue à lire. 
        Le confinement d'avril ainsi que les cas de covid à EPITA nous ont empêché de nous voir et ont donc ralenti le rythme de travail, car 
        il est bien plus difficile pour ce type de projet de travailler à distance qu'en groupe.
        Nous avons aussi eu de très gros conflits de merge quatre jours avant le rendu, après une validation sans vérification d'une pull 
        request problématique. Il nous a fallu sept heures et 17 commits avant de résoudre les problèmes, car il nous a fallu récupérer les 
        fichiers importants de la PR et revert les fichiers problématiques.
    \newpage
    