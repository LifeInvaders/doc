\subsection{Impressions des membres de l'équipe}

    \subsubsection{Développement du jeu}

        Le développement de ce jeu s'est avéré être un vrai défi technique, avec de nouvelles compétences à apprendre. 
        Le logiciel Unity est très riche en fonctionnalités, et de ce fait assez complexe à utiliser de par son interface 
        recouverte d'options. Cependant, passée cette appréhension, cet éditeur se montre pratique pour certaines tâches répétitives, et 
        on s'habitue vite à son utilisation. Nous avons développé les scripts en C\# sur Jetbrains Rider, qui est un outil 
        excellent, d'une simplicité d'utilisation remarquable et incluant des outils très utiles comme git. 


    \subsubsection{Gestion du planning et des tâches}

        Le planning était assez difficile à tenir, car nous avions prévu d'avoir fini 80\% du jeu à la deuxième soutenance. 
        Cependant, il s'agissait d'un choix du groupe, et cela nous a permis de ne pas nous laisser submerger par le travail 
        à finir. 