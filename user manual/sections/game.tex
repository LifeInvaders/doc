\subsection{Objectifs}

    \paragraph{Objectifs}

        L'objectif du jeu est de traquer sa cible, pointée par la boussole en bas à droite de l'écran, et l'éliminer. Cependant, 
        vous êtes vous aussi la cible d'un joueur, et devez donc fuir votre agresseur. Quatre pouvoirs sont mis à votre disposition : 
        choisissez celui qui vous convient le mieux, afin de prendre l'avantage sur les autres joueurs.


    \paragraph{Système de points}

        Les points sont calculés en fonction de vos assassinats : tuer sa cible rapporte des points, l'empoisonner en rapporte plus, 
        mais tuer un personnage non-joueur en fait perdre, et tuer la cible de quelqu'un d'autre fait gagner des points à cette personne. 
        A la fin de la partie, les trois premiers joueurs sont affichés, ainsi qu'un tableau des scores.
            
        
\subsection{Contrôles et menu}


    \paragraph{Contrôles}

        Les contrôles de jeu, modifiables depuis le menu paramètres, sont par défaut Z/Q/S/D pour 
        avancer/gauche/reculer/droite, et espace pour sauter. La direction du personnage se fait à la souris. 
        Pour éliminer une cible, vous devez d'abord passer en mode verouillage (par défaut : clic molette), pour mettre en surbrillance 
        votre cible. Utilisez le clic droit pour la verouiller, et le clic gauche pour l'éliminer lorssqu'elle se trouve à portée.


    \paragraph{Tableau des scores}

        Le tableau des scores est affichable avec la touche de tabulation (Tab). Celui-ci permet d’avoir le classement des joueurs selon 
        leur score actuel. Chaque ligne est de la forme :[Nom du joueur] [Score] [Assassinats / morts].

\subsection{Pouvoirs}

    Le jeu propose différents pouvoirs, séléctionnables à chaque mort et utilisables en jeu grâce à la touche [E] par défaut. 
    Voici la liste des pouvoirs et leur description :

    \paragraph{Bombe de fumée}

        La bombe de fumée est, comme son nom tend à l'indiquer, un objet qui active des particules autour d'un joueur lui 
        permettant de créer une confusion afin de se retirer ou d'attaquer discrètement. Son grand intérêt réside dans le fait 
        qu'elle paralyse tous les personnages, NPC comme joueurs (sauf le lanceur, bien évidemment) jusqu'à dissipation de la fumée. 
        C'est donc un atout majeur de fuite plus que d'attaque.

    \paragraph{Poison}

        Le poison n'est pas vraiment un atout en jeu, car il nécéssite de suivre sa victime et de la tuer comme pour une 
        élimination classique. Cependant, lors de l'élimination de la cible, cette dernière donne plus de points, car son 
        élimination est discrète.

    \paragraph{Lancer de couteaux}

        Le pouvoir du lancer de couteaux est de loin celui ayant été le plus long à mettre en oeuvre. Lorsque l'assassin 
        verrouille sa cible, il peut lui lancer un couteau jusqu'à 15 mètres (contre 2 mètres pour les attaques classiques).
        La cible touchée se met à boiter, voit sa vitesse réduite et ne peut ni courir, ni emprunter d'échelles pendant une certaine 
        période donnant à son assaillant le temps de l'achever. 

    \paragraph{Sprint}
	
        Le pouvoir de sprint permet au joueur de se déplacer très rapidement et d'être invulnérable face aux attaques 
        au couteau. En revanche, le joueur ne peut pas emprunter d'échelles ou de tyroliennes en même temps que ce pouvoir, afin 
        de ne pas déséquilibrer le jeu. Il laisse également une longue traînée de fumée derrière lui, qui lui fait perdre 
        tout semblant de discrétion. 

    
\subsection{Personnages non-joueurs}

        Les personnages non-joueurs (PNJ, ou NPC en anglais pour Non-playable Characters) sont des personnages contrôlés par une intelligence 
        artificielle. Ils se répartissent et se déplacent aléatoirement sur la carte, permettant au joueur de se déplacer discrètement sur la carte au milieu des 
        autres personnages.
    