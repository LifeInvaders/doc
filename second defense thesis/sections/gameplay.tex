\subsection{Gameplay}

\subsubsection{Système de verrouillage}

Pour tuer un personnage (joueur ou NPC),
nous avons ajouté un système de verrouillage. Celui-ci est assez pratique, lorsque l'on passe dans ce mode, l'écran change,
un effet réalisé grâce au post processing de Unity permet de donner un effet sépia/vieux films\footnote{voir Effets Graphiques}.
Un contour blanc autour des personnages visés au centre de l'écran permettent voir quel personnage va être sélectionné.

Une fois sélectionné et lorsque le joueur est proche de sa cible, il peut alors l'éliminer.

L'effet a demandé de créer plusieurs overlay de caméra, afin d'avoir un effet graphique
appliqué uniquement sur certains layers, et de les surperposer les unes sur les autres. 

Nous avons fourni un vrai travail sur les différentes animations de morts des personnages.
Lors de sa réalisation, un problème s'est posé : Il fallait que les animations de deux GameObject (ici le joueur qui tue et le NPC/joueur tué) 
soient synchronisées et très précises spatialement. Après avoir regardé plusieurs types de solutions, nous nous sommes tournés vers un outil préintégré nommé Timeline.
Ce dernier permet de réaliser des clips vidéos.

Voici donc comment nous avons intégré les timelines :
\newline
Des faux personnages jouant les animations sont ajoutés sur la cartes à la position et rotation du tueur.
On masque le tueur et le tué de la carte.
On change le mesh et le matériau de chaque personnage de la timeline pour qu'il corresponde au tueur et au
personnage tué.

Une fois l'animation terminé, un signal est envoyé à script qui réaffiche alors le joueur qui était masqué.

Le personnage tué disparait de la carte au bout d'une dizaine de secondes avec un shader fait avec Shader Graph.

\subsubsection{}