\subsection{Intelligence Artificielle}
    \subsubsection{NPC Zone \& Système de déplacement}
    Nous avons essayé plusieurs types de déplacement pour les IA.
    Nous avons créé des NPC zones qui permettent de créer des quartiers
    où les NPC peuvent se balader librement dans la zone.
    Si un NPC est tué dans la zone, un nouveau apparaît avec un effet visuel approprié.

    Pour le système de déplacement de l'IA,
    nous avons essayé plusieurs algorithmes différents :\newline
    
    \begin{itemize}
        \item Le premier algorithme que nous avions montré lors de la première soutenance
        était un déplacement vers un élément alétoire d'une liste de coordonées prédéfinie.
        Le tracé était vu et revu, et de ce fait, trop prévisible.
        De plus, si un personnage allait d'un point A vers un point B, et un autre de B vers A,
        alors ils allaient prendre le même chemin et donc se croiser (face à face) et finir coincé.
        \newline
        \item  Nous avons donc décidé de rajouter plus d'alétoire dans leurs déplacements.
        Les IA cherchent une destination "accessible" dans un rayon proche.
        Le problème était que la librairie Unity AI intégrée cherche un chemin complet ou partiel.
        Le rendu final montrait des personnages qui finissaient toujours par être attiré par les bordures de la carte.
        \newline

        \item    Le dernier système intégré changait un point.
        Au lieu de chercher une destination proche de lui, il cherchait une destination dans le rayon
        d'action de la NPC zone.
    
    \end{itemize}
    
   
    
    \subsubsection{Evenements}

    Afin de rendre les IA moins scriptées, nous avons ajouté des événements alétoire.
    
    Il y a des zones de discussions où les NPC peuvent aller.
    Les joueurs peuvent interagir avec et se fondre dans la discussion, pour se cacher.

    % TODO : FINISH THIS PART