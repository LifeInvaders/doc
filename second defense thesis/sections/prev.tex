\subsection{Avances et prévisions}
\subsubsection{Avances}
\begin{table}[!hbt]
    \begin{center}
        \begin{tabular}{l|ll}
            \rowcolor[HTML]{000000} 
            {\color[HTML]{FFFFFF} \backslashbox{\textbf{Partie}}{\textbf{Tâche}}} & {\color[HTML]{FFFFFF} \textbf{Prévu}} & {\color[HTML]{FFFFFF} \textbf{Réalisé}} \\
            \rowcolor[HTML]{FFFFFF} 
            \textbf{Mouvement}                         & 70\%                                  & \cellcolor[HTML]{68CBD0}70\%         \\
            \rowcolor[HTML]{C0C0C0} 
            \textbf{Interface/HUD}                     & 60\%                                  & \cellcolor[HTML]{FD6864}50\%         \\
            \textbf{Cartes}                            & 80\%                                  & \cellcolor[HTML]{68CBD0}90\%         \\
            \rowcolor[HTML]{C0C0C0}
			\textbf{Réseau}    						   & 50\%          						   & \cellcolor[HTML]{FFCC67}70\%         \\
            \textbf{IA}                                & 60\%                                  & \cellcolor[HTML]{FFCC67}80\%         \\
            \rowcolor[HTML]{C0C0C0} 
            \textbf{Mécaniques de jeu}                 & 70\%                                  & \cellcolor[HTML]{FFCC67}80\%         \\
            \textbf{Progression}                       & 60\%                                  & \cellcolor[HTML]{FD6864}50\%        
            \end{tabular}
    \end{center}
    \caption{Tableau des avances et retards dans les différentes parties}
\end{table}
\subsubsection{Prévisions}
Le jeu est à présent dans un état "jouable". Cependant, il faut encore implémenter de nombreux systèmes pour finaliser notre vision pour le projet.

	\subsubsection{Multiplayer}
	Certains systèmes ont été créés, comme les finishers ou les pouvoirs, mais n'ont pas encore été 
	implémentés en multijoueur. L'objectif pour la prochaine soutenance sera donc d'intégrer les mécaniques 
	de jeu qui ne l'ont pas encore été au réseau Photon pour permettre au joueurs d'y accéder. De plus, le 
	travail continue sur la synchronisation ie. des personnages joueurs et non-joueurs, notamment il est 
	question d'optimisation pour le mouvement des IA qui génère encore une utilisation de bande passante importante. 
	Il serait intéressant de tenter l'utilisation d'une méthode de compression pour les vecteurs afin de diminuer 
	la quantité d'information envoyée.

	\subsubsection{Progression}
		Bien qu'un système d' expérience et de niveau ait été implémenté, celui-ci n'a pour l'instant pour 
		l'instant pas d'utilité. Mais de nombreuses opportunités de progression se sont ouvertes, 
		notamment grâce au travail de Dov. Par exemple, nous comptons permettre le débloquage des 
		finishers grâce à ce système de progression. Nous pourrons également modifier le système actuel 
		de customisation de personnage et restreindre l'utilisation de certaines options d'apparence à un certain niveau.

