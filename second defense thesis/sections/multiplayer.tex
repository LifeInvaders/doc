\subsection{Réseau}

    \subsubsection{Transition du Lobby vers le jeu}
        Un système a été mis en place permettant le passage de la scène de lobby vers la scène de jeu.
        Ce système fonctionne a l'aide d'un timer, qui est synchronisé entre tout les joueurs, y compris ceux
        rejoignant le lobby après son lancement. Ce timer ne se lance uniquement après que le serveur soit à moitié
        rempli et dure 3 minutes. Si le nombre de joueurs descend en dessous de ce seuil, le timer s'arrête. De plus, qaund le serveur
        est complet, le temps d'attente est réduit à 30 secondes.
        A la fin du timer, le Master Client charge la map de jeu, qui est synchronisé avec tout les joueurs.
    
    \subsubsection{Synchronisation des joueurs}
        Tout comme sur le lobby, les mouvements des joueurs ainsi que leurs animations sont synchronisés. Cependant,
        un deuxième élément s'ajoute à celà: l'apparition (spawn) des joueurs. Pour celà, des points d'apparitions (spawpoints)
        sont répartis sur la map. Le Master Client distribue ces points aux joueurs qui apparaîtront à l'endroit reçu. Celà permet
        de faire apparaîte chaque joueur à une position unique sur la carte : un point = un joueur, pas plus. Une fois celà fait,
        il faut également synhcroniser la réapparition des joueurs, pour celà on applique le même système, en ne prenant en compte que les joueurs
        morts.

    \subsubsection{Synchronisation de l'IA}
        La grande difficulté niveau multijoueur, c'est la synchronisation des PNJSCelà est une tâche important dû à la nature du jeu, 
        mais également difficile dû au grand nombre d'IA présentes sur la map.

        Tout d'abord, il faut synchroniser l'apparition des PNJs. Encore une fois, c'est le Master Client qui instancie les personnages
        grâce à Photon. Ils sont donc au départ placé de la même façon pour tout les joueurs. Il faut également synchroniser leur apparence.
        Encore une fois, cette dernière est déterminé par le Master Client puis partagé aux autres joueurs grâce à une méthode RPC.

        Mais les problèmes commencent au moment de synchroniser le mouvement des IA. Le système qui a été créé par Dov permet
        à l'IA de se déplacer sur la map, il faut maintenant que ce mouvement soit propagé de façon quasi-identique à tout les joueurs.
        Pour celà, plusieurs méthodes ont été envisagés:

            -L'utilisation de Photon Transform View, comme pour les joueurs. Ce système a vite montré ses limites, car inadapté à la synchronisation
            d'un grand nombre d'objets, fonctionant sur la base d'un envoi pseudo-continu d'informations. Ainsi de nombreux problèmes apparaissaient,
            et la synchronisation des mouvement en a souffert. 

            -Calcul de chemin client-side à partir du même point. L'idée est la suivante: le master client calcule un point, qui est la destination
            de l'IA et la partage aux autres joueurs. Puis chaque joueur calcule le chemin pris par l'IA pour y arriver. Celà réduit considérablement la
            quantité d'information échangée, mais un autre problème se pose: le calcul de chemin pour les NavMesh Agents n'est pas déterministe. Ainsi le
            chemin calculé par chaque client à partir du même point n'est pas le même, ce qui entraîne également une désynchronisation de la position.

            -Enfin, le choix retenu est le calcul d'un chemin entier par le Master Client, qui envoie ensuite l'intégralité de ce chemin aux autres joueurs.
            Ainsi le chemin est le même pour tout le monde, mais l'envoi des points se fait de façon discrète: on envoit uniquement l'array de positions généré
            par le Master Client. Ce système permet d'avoir une synchronisation satisfaisante des déplacement et une utilisation minime de la bande passante.

    \subsubsection{Synchronisation des assassinats et des morts}
        Une dernière partie de la synchronisation inclue celle des évènement de mort, pour les joueurs ainsi
        que les IA. Pour celà, il a été décidé d'utiliser le sytème d'Event Photon. Auparavant, la synchronisation de méthodes
        passait par l'utilisation de Photon View et de RPC. Mais le système de kill/death demande une propagation plus importante
        de l'évènement, car plusieurs systèmes différents doivent y réagir. C'est la que Photon rentre en jeu.

        En effet, ce dernier permet de synchroniser le déclenchement d'évènement. Pour celà, on utilise la méthode RaiseEvent ainsi
        qu'un code représentant notre évènement. Cette action appelle la méthode callback OnEvent, dans laquel il suffit d'associer le
        code de l'évènement à une méthode. Ainsi, à la mort d'un joueur, le tueur déclenche l'évènement mort en indiquant l'identité du
        joueur tué. Chaque joueur reçoit cette évènement et peut déterminer, par exemple, si le joueur mort est lui-même, ou bien si un
        autre joueur a tué sa cible... En bref, chaque client prend une décision en fonction de l'information reçu
        Les détails du système de morts sont dans la partie gameplay.

    \subsubsection{Synchronisation du déroulé de partie}
        De la même manière, un système d'event a été mis en place permettant de synchroniser les différentes étapes du jeu.

