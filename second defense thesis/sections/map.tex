\subsection{Carte du jeu}

La carte est désormais finie, et de nouveaux éléments ont été ajoutés, comme un shader animé (créé avec l'outil Shader Graph) pour l'eau,
faite par Dov, ou encore de nouvelles lumières dynamiques. Elle est aussi dotée de nombreuses échelles, qui permettent de fuir ses poursuivants 
de façon discrète, ainsi que de venelles reliant les avenues.
En outre, les nombreux NPC ainsi que les marchandises exposées au milieu des rues font aussi de bonnes diversions.
Enfin, l'ajout d'escaliers offrant un second accès à la colline permettent non seulement de désengorger la butte, envahie par les NPC, mais aussi 
de redynamiser la digue qui était jusque là exempte de tout intérêt : pas de bâtiments, pas de cachettes...

Mais la principale nouveauté est le mode nuit : en effet, il est désormais possible de passer du jour à la nuit grâce à de simples boutons-radios. 
Le mode nuit applique des effets de post-processing à toutes les textures de la carte, et assombrissant les couleurs et en appliquant certains effets 
visuels se traduisant en jeu par un environnement plus sombre (deonc nocturne). 
Ce mode nuit permet de faire ressortir la beauté de la ville endormie, tout en ajoutant un côté angoissant aux parties,
qui deviennent \textit{de facto} beaucoup plus animées.



