\section{Prévisions}

\subsection{Map}
    Afin d'améliorer l'expérience des joueurs, nous réfléchissons actuellement à la création d'une 
    seconde map, une fois la première peaufinée.
\subsection{Interface}
    Sur l’interface, nous prévoyons d’intégrer la catégorie « Liste des serveurs » au menu Multijoueur. 
    Les joueurs pourront ainsi choisir librement le serveur qu’ils souhaitent rejoindre. 
    Ce menu permettrait aussi aux joueurs souhaitant s'amuser en petit comité de créer des 
    parties privées grâce à un onglet "Créer un serveur" à partir de paramètres personnalisables. 
    Il sera également possible de connaître les noms des joueurs connectés à la partie en pressant 
    la touche TAB grâce à un menu présent en jeu.

\subsection{Réseau}
    Nous souhaitons ajouter un chat textuel dans le lobby et au sein des parties mêmes afin d’augmenter la sociabilité et l’adversité entre les joueurs.

\subsection{I.A.}
    Pour la seconde période, notre objectif serait de pouvoir avoir une IA plus autonome et intelligente, qui n'emprunte plus forcément le chemin le plus court. 
    De plus, les joueurs peuvent actuellement bousculer les NPC qui s'arrêtent alors de marcher, mais ils ne peuvent jamais les pousser; 
    ceci sera donc à paufiner.

\subsection{Mécaniques de jeu}
    Nous prévoyons d'intégrer pour la deuxième soutenance des pouvoirs
    comme un écran de fumée ou encore des couteaux,
    pour rendre le jeu plus intéréssant et dynamique.\\
    Nous comptons également implémenter le script de choix de la cible.
    Même si celui-ci a été réalisé il n'est pas encore utilisé en jeu.
    Il faudra principalement donner un sens à l'attribution des cibles,
    sous la forme de points octroyés au joueur.

\subsection{Autres}
    Pour tester le multijoueur avec des invités, nous avons réalisé 
    un début de launcher qui télécharge les dernières mises à jour et les installe.
    Nous aimerions continuer à le développer pour avoir un lanceur
    qui se mette à jour automatiquement sans à avoir à le réinstaller.
