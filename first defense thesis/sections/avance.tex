\section{Avances et retards}


   
\begin{table}[!hbt]
    \begin{center}
        \begin{tabular}{l|ll}
            \rowcolor[HTML]{000000} 
            {\color[HTML]{FFFFFF} \backslashbox{\textbf{Partie}}{\textbf{Tâche}}} & {\color[HTML]{FFFFFF} \textbf{Prévu}} & {\color[HTML]{FFFFFF} \textbf{Réalisé}} \\
            \rowcolor[HTML]{FFFFFF} 
            \textbf{Mouvement}                         & 50\%                                  & \cellcolor[HTML]{FFCC67}80\%         \\
            \rowcolor[HTML]{C0C0C0} 
            \textbf{Interface/HUD}                    & 40\%                                  & \cellcolor[HTML]{68CBD0}40\%         \\
            \textbf{Cartes}                            & 40\%                                  & \cellcolor[HTML]{68CBD0}40\%         \\
            \cellcolor[HTML]{C0C0C0}\textbf{Réseau}    & \cellcolor[HTML]{C0C0C0}20\%          & \cellcolor[HTML]{FFCC67}50\%         \\
            \textbf{IA}                                & 30\%                                  & \cellcolor[HTML]{FFCC67}40\%         \\
            \rowcolor[HTML]{C0C0C0} 
            \textbf{Mécaniques de jeu}                 & 50\%                                  & \cellcolor[HTML]{FD6864}40\%         \\
            \textbf{Progression}                       & 20\%                                  & \cellcolor[HTML]{FFCC67}30\%        
            \end{tabular}
    \end{center}
    \caption{Tableau des avances et retards dans les différentes parties}
\end{table}


\subsection{Les réussites}

La création du système de mouvement s'est avérée moins compliquée que nous l'avions prévu.
Il manque un peu de finesse, et une physique de saut plus réaliste ne serait pas de trop non plus, mais il reste assez complet.
L'utilisation du pathfinding pour les mouvements des IA nous a elle aussi étonné par sa simplicité.
Un lobby fonctionnel pour le multijoueur a aussi été réalisé ; cela nous permet non seulement de continuer sereinement la suite du projet, 
notamment concernant le lancement d'une partie multijoueur, mais également d'acquérir des connaisances sur Unity et sur Photon qui nous 
permettront d'arriver au bout de ce projet.




\subsection{Les retards}

Certaines mécaniques de jeu sont encore en développement et ne sont pas finalisées (réapparition, système de points...)
D'autres, comme l'attribution des cibles, sont finalisées ou presque mais ne peuvent pas encore être implémentées.
    
