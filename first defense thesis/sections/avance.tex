\section{Avances et retards}


   
\begin{table}[!hbt]
    \begin{center}
        \begin{tabular}{l|ll}
            \rowcolor[HTML]{000000} 
            {\color[HTML]{FFFFFF} \backslashbox{\textbf{Partie}}{\textbf{Tâche}}} & {\color[HTML]{FFFFFF} \textbf{Prévu}} & {\color[HTML]{FFFFFF} \textbf{Réalisé}} \\
            \rowcolor[HTML]{FFFFFF} 
            \textbf{Mouvement}                         & 50\%                                  & \cellcolor[HTML]{FFCC67}80\%         \\
            \rowcolor[HTML]{C0C0C0} 
            \textbf{Interface/HUD}                    & 40\%                                  & \cellcolor[HTML]{68CBD0}40\%         \\
            \textbf{Cartes}                            & 40\%                                  & \cellcolor[HTML]{68CBD0}40\%         \\
            \cellcolor[HTML]{C0C0C0}\textbf{Réseau}    & \cellcolor[HTML]{C0C0C0}20\%          & \cellcolor[HTML]{FFCC67}50\%         \\
            \textbf{IA}                                & 30\%                                  & \cellcolor[HTML]{FFCC67}40\%         \\
            \rowcolor[HTML]{C0C0C0} 
            \textbf{Mécaniques de jeu}                 & 50\%                                  & \cellcolor[HTML]{FD6864}40\%         \\
            \textbf{Progression}                       & 20\%                                  & \cellcolor[HTML]{FFCC67}30\%        
            \end{tabular}
    \end{center}
    \caption{Tableau des avances et retards dans les différentes parties}
\end{table}


\subsection{Les réussites}

La création du système de mouvement n'a pas été finalement très compliqué.
Il manque un peu de finesse, et d'une physique de saut plus réaliste. Il reste cepandant complet.

L'utilisation du pathfinding pour les mouvements des IA a été beaucoup plus simple à implémenter que prévu.

La réalisation d'un lobby fonctionnel pour le multijoueur a été accomplie. Celà nous permet non seulement de continuer serainement la suite du projet, notamment le lancement d'une partie multijoueur, mais il a également permit d'acquérir des connaisances sur Unity et sur Photon qui augmentent les chances de réussites pour la suite du projet.




\subsection{Les retards}

Certaines mécaniques de jeu sont encore en dévellopement et ne sont pas finalisées.
D'autres sont finalisés ou presque mais ne peuvent pas encore être implémentés.
    
