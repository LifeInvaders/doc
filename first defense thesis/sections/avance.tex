\documentclass[../doc.tex]{subfile}
\section{Avances et retards}


   
\begin{table}[!hbt]
    \begin{center}
        \begin{tabular}{l|ll}
            \rowcolor[HTML]{000000} 
            {\color[HTML]{FFFFFF} \backslashbox{\textbf{Partie}}{\textbf{Tâche}}} & {\color[HTML]{FFFFFF} \textbf{Prévu}} & {\color[HTML]{FFFFFF} \textbf{Réalisé}} \\
            \rowcolor[HTML]{FFFFFF} 
            \textbf{Mouvement}                         & 50\%                                  & \cellcolor[HTML]{FFCC67}80\%         \\
            \rowcolor[HTML]{C0C0C0} 
            \textbf{Interface/HUD}                    & 40\%                                  & \cellcolor[HTML]{68CBD0}40\%         \\
            \textbf{Cartes}                            & 40\%                                  & \cellcolor[HTML]{68CBD0}40\%         \\
            \cellcolor[HTML]{C0C0C0}\textbf{Réseau}    & \cellcolor[HTML]{C0C0C0}20\%          & \cellcolor[HTML]{FFCC67}50\%         \\
            \textbf{IA}                                & 30\%                                  & \cellcolor[HTML]{FFCC67}40\%         \\
            \rowcolor[HTML]{C0C0C0} 
            \textbf{Mécaniques de jeu}                 & 50\%                                  & \cellcolor[HTML]{FD6864}40\%         \\
            \textbf{Progression}                       & 20\%                                  & \cellcolor[HTML]{68CBD0}20\%        
            \end{tabular}
    \end{center}
    \caption{Tableau des avances et retards dans les différentes parties}
\end{table}


\subsection{Les réussites}

La création du système de mouvement n'a pas été finalement très compliqué.
Il manque un peu de finesse, et d'une physique de saut plus réaliste. Il reste cepandant complet.

L'utilisation du pathfinding pour les mouvements des IA a été beaucoup plus simple à implémenter que prévu.




\subsection{Les retards}

Certaines mécaniques de jeu sont encore en dévellopement et ne sont pas finalisées.
    
