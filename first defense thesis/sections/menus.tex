\documentclass[../doc.tex]{subfiles}

\begin{document}

Les interfaces sont une partie importante de tout jeu. Elles permettent de donner au joueur plus de contrôle sur son expérience et de présenter les informations nécessaires de façon clair, facilement accessible et si possible esthétique.
Un progrès important a été fait quant à la création d'interfaces.

\subsection{Menu d'accueil}
    Le menu d'accueil est la première chose que le joueur voit en lançant le jeu.
    Celui-ci est actuellement entièrement implémenté et fonctionnel. Cependant il continuera d'évoluer au fil du projet et de ses besoins.
    
    \subsubsection{Thème général}
    Le thème du menu principal doit être représentatif du jeu. Il est donc évident que ce dernier affiche la couleur dés le lancement: ceci est un jeu de pirates.
    Le background, les couleurs, la musique... Y est également présent le logo du groupe ainsi que celui du jeu. C'est Harrys qui a réalisé ce style pirate. 
    
    \subsubsection{Fonctionnalités}
    Plusieurs options sont disponibles. Le joueur peut tout d'abord se connecter au serveur Photon. Il accède ainsi à un menu lui permettant de rejoindre une salle. Il peut également y modifier son pseudo, qui sera affiché aux autres joueurs. Seule la fonction "Quick Play" a été implémenté(cf. multijoueur) mais l'objectif est d'implémenter une liste ainsi qu'un menu de création de salles. Cette partie du menu principal a été réalisée par Julien.
    
    Du menu principal, on peut également modifier les paramètres de jeu comme la résolution, le mode plein écran, le volume... De plus on peut modifier les touches du jeu. Ces changements sont alors sauvegardées (cf. système de sauvegarde).
    
    Un menu de customisation de personnage a également été implémenté (cf. système de progression). On peut y modifier certaines caractéristiques, comme la couleur des vêtements ou de la peau. Un aperçu du modèle est disponible (et peut pivoter grâce à la souris). Les paramètres, ainsi que ce menu, ont été réalisés par Harrys.
    
    Enfin, on peut y fermer l'application.
    
    \begin{figure}[hbt!]
                \centering
                \captionsetup{justification=centering}
                \begin{subfigure}[b]{0.3\textwidth}
                    \includegraphics[scale=0.1]{mainmenu.png} 
                    \caption{Menu principal}
                \end{subfigure}
                \hspace{150pt}
                \begin{subfigure}[b]{0.3\textwidth}
                    \includegraphics[scale=0.1]{settingsmenu.jpg} 
                    \caption{Paramètres}
                \end{subfigure}
                \begin{subfigure}[b]{0.3\textwidth}
                    \includegraphics[scale=0.1]{charcustom.jpg} 
                    \caption{Apparence}
                \end{subfigure}
                \begin{subfigure}[b]{0.3\textwidth}
                    \includegraphics[scale=0.1]{photonmenu.png} 
                    \caption{Menu multijoueur}
                \end{subfigure}
                \caption{Différents affichages du menu d'accueil}
    \end{figure}

\end{document}
